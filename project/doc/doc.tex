\documentclass{article}
\usepackage{graphicx}

\usepackage[utf8]{inputenc}
\usepackage[a4paper]{geometry}
\usepackage{etoolbox}
\usepackage{amsmath}
\usepackage{stmaryrd}
\usepackage{tikz}


\makeatletter
\patchcmd{\maketitle}{\@fnsymbol}{\@alph}{}{}  % Footnote numbers from symbols to small letters
\makeatother

\title{Assignments 6.1\\ \large{Modern Databases UvA-2015}}
\author{Abe Wiersma\\Student number: 10433120}

\date{\today}

\begin{document}

\maketitle

\section*{Differences between a B Tree and a B+ Tree}
Two major differences between B Trees and B+ Trees can be found:
\begin{itemize}
    \item B Trees store data in their internal nodes, where B+ Trees store
    data only in their leaves(the outer most nodes.)
    \item The leaves of a B+ tree are also a linked list making iteration over
    the keys and data of a B+ tree a lot easier. In a B Tree you would always
    have to do a depth first search to get all keys and data.
\end{itemize}

\section*{Operations in our B+ tree}
\begin{itemize}
    \item Searching for a key.
    \begin{enumerate}
        \item Start with root node.
        \item Loop over keys of current node until a key lower than the key
        being searched for is found.
        \item If None is found raise a KeyError, else go one level down into
        the key found.
        \item If current node is a leaf try to find the key directly in the
        bucket, raise KeyError when the key is not found whilst in a leaf.
        If the current node is a regular node Start with step 2 again.
    \end{enumerate}
    \item Inserting a key, value pair.
    \begin{enumerate}
        \item Using part of the above algorithm find the correct leaf, set the
        key in the bucket of the leave found to value.
        \item If the leaf has grown beyond $max\_size$ split up the leaf into two
        leaves, with equally sized buckets.
        \item This change moves up the tree checking whether every parent root
        also needs to be split up.
    \end{enumerate}
    \item Deleting a key and it's data.
    \begin{enumerate}
        \item Using the search algorithm find the the key and remove it.
        \item If a leaf has shrunk below half of $max\_size$ find it's
        neighbors.
        \item If there is a neighbor with whom the current node can merge,
        merge! When a merge occurs it can happen that this merge moves up the
        tree. In this process a node merges it's buckets together with it's
        neighbor.
        \item If step 3 failed check if the neighbors have at least
        $(max\_size/2) + 1$. If so snatch away one of their key value pairs.

    \end{enumerate}
\end{itemize}

\section*{Bulk loading into a B+ tree}
Bulk loading happens when a lot of documents have to be added into an empty
tree. You could start with an empty tree and use the standard insert, but this
is slow in comparison with an optimized bulk-load. Instead one can pre-sort the
documents and then construct a maximally utilized B+ tree from these pre-sorted
documents.

\section*{Inserting keys 32 and 42}
\begin{tikzpicture}[node distance=1cm, scale=1, line width=0.1em,
    rectangle split, rectangle split horizontal, rectangle split parts=3]
\node[name=B1, draw] {
    $2$
    \nodepart{second} $5$
    \nodepart{third} $6$
};
\node[name=B2, draw, right=of B1] {
    $21$
    \nodepart{second} $27$
    \nodepart{third} {\ \ }
};
\node[name=B3, draw, right=of B2] {
    $37$
    \nodepart{second} $39$
    \nodepart{third} $44$
};
\node[name=B4, draw, right=of B3] {
    $49$
    \nodepart{second} $53$
    \nodepart{third} $59$
};
\path (B2) -- node (BC) {} (B3); 
\node[name=A1, draw, above=of BC] {
    $21$
    \nodepart{second} $37$
    \nodepart{third} $49$
};
\path[draw, ->] (A1.south west) -- (B1.north);
\path[draw, ->] (A1.one split south) -- (B2.north);
\path[draw, ->] (A1.two split south) -- (B3.north);
\path[draw, ->] (A1.south east) -- (B4.north);
\end{tikzpicture}

Insert the 32.
\begin{tikzpicture}[node distance=1cm, scale=1, line width=0.1em,
    rectangle split, rectangle split horizontal, rectangle split parts=3]
\node[name=B1, draw] {
    $2$
    \nodepart{second} $5$
    \nodepart{third} $6$
};
\node[name=B2, draw, right=of B1] {
    $21$
    \nodepart{second} $27$
    \nodepart{third} $32$
};
\node[name=B3, draw, right=of B2] {
    $37$
    \nodepart{second} $39$
    \nodepart{third} $44$
};
\node[name=B4, draw, right=of B3] {
    $49$
    \nodepart{second} $53$
    \nodepart{third} $59$
};
\path (B2) -- node (BC) {} (B3); 
\node[name=A1, draw, above=of BC] {
    $21$
    \nodepart{second} $37$
    \nodepart{third} $49$
};
\path[draw, ->] (A1.south west) -- (B1.north);
\path[draw, ->] (A1.one split south) -- (B2.north);
\path[draw, ->] (A1.two split south) -- (B3.north);
\path[draw, ->] (A1.south east) -- (B4.north);
\end{tikzpicture}

Compiled niet, HIER MOET DIE VAN 42 nog BIJ!

\section*{Removing keys 49, 59 and 44}
Hier het remove gedeelte.
\end{document}
